\documentclass[12pt,a4paper]{article}
\usepackage[utf8]{inputenc}
%\usepackage[hidelinks]{hyperref}
\usepackage{hyperref}
\usepackage{xcolor}
\usepackage{cite}
\usepackage[version=4]{mhchem} % for isotope symbols
\usepackage{a4wide}
\usepackage[lmargin=1cm,rmargin=1cm,tmargin=1cm,bmargin=2cm,headheight=0in]{geometry}
\usepackage{amssymb}


\hypersetup{colorlinks, linkcolor={blue!50!black}, citecolor={blue!50!black},
	urlcolor={blue!50!black}}

\begin{document}
\title{Journal Entry - Week Five - Entry Four}
\author{Caspian Nicholls, u1027945}
\date{\today}

\maketitle

\noindent
I give permission for portions of this work to be used as examples in future science communication course notes.
% \textit{[Delete if you don't want to do this]}\\

\section*{Summary of response to peer feedback}

First and foremost, I tried to remove or clear up any jargon that I used and make my writing more concise, either by restructuring sentences or using bullet points.
I made sure to define terms (especially \textit{octupole correlations}) that the reader might not be familiar with the first time I mentioned them, as this was something my peer marker picked up on in a few different places.
I also split up or restructured some of my paragraphs to make sure that they each (where possible) focus on a single idea.
Finally, I deleted some sentences that did not add anything significant to this journal entry.

\bigskip
\noindent
\textbf{Article title:} Search for octupole correlations in neutron-rich $\ce{^{148}Ce}$ nucleus \\
\textbf{Authors:} Y. J. Chen, S. J. Zhu, J. H. Hamilton, A. V. Ramayya, J. K. Hwang, M. Sakhaee, Y. X. Luo, J. O. Rasmussen, K. Li, I. Y. Lee, X. L. Che, H. B. Ding, and M. L. Li \\
\textbf{Reference:} Phys. Rev. C 73, 054316, May 2006. \\
\textbf{Link:}  \href{10.1103/PhysRevC.73.0543 at different energies.16}{Article abstract}. \\
\textbf{Word count:} 743.
%\textit{Give the word count of everything below this point, e.g. copy paste into the \href{https://app.uio.no/ifi/texcount/online.php}{TeXcount} online tool. 750 word limit.}

\section*{Context}
%Briefly summarise the background of the area of physics relevant to the problem they are solving. Why it is an important or relevant area and problem?
Interactions between nucleons within nuclei can affect nuclear shapes.
One such interaction is that of octupole correlations, the observation of which can provide evidence that a nucleus possesses a non-spherical (often pear-shaped) charge distribution.
These correlations arise when nucleons in nuclear orbitals with opposite parity (positive and negative) and a precise relationship between their quantum numbers interact strongly.
Specifically, the interacting orbitals will satisfy $\Delta n = 1$ and $\Delta l = \Delta j = 3$, where $n$, $l$ and $j$ denote the principal, orbital angular momentum and total angular momentum quantum numbers.

\medskip
Theoretical calculations that utilise accepted models of nuclear structure predict that several nuclides with proton number $Z \sim 56$ and neutron number $N \sim 88$ should exhibit these correlations.
Yet, experimental evidence of octupole correlations has not been recorded for many of these nuclides, prompting further research.
Ultimately, observation of these correlations can provide useful information about the energy level structure and shape of nuclei with mass number $A\sim144$.


\section*{Purpose}
%Explain what problem they are solving, or what information they were trying to gain.
In this work, experiments are performed to hunt for signs of octupole correlations in $\ce{^{148}Ce}$, which with $Z = 58$ and $N = 90$, should display this type of interaction. The authors also seek to extend the existing picture (\textit{level scheme}) of the excited states of this nuclide.


\section*{Approach}
%Describe explicitly the method they use to complete their work, i.e. the experimental system and measurements, the model design and computational techniques used, or the setup of their equations and the assumptions applied.
A sample of $\ce{^{252}Cf}$ was placed inside a gamma ray detector array, the spontaneous fission of which produced (among other isotopes) $\ce{^{148}Ce}$.
Then, gamma rays that were emitted from excited states of the radioisotopes produced by fission events were measured by the detector array.
Signals from sources other than $\ce{^{148}Ce}$ were filtered out during the analysis phase.

\medskip
To build upon the existing level scheme for $\ce{^{148}Ce}$, gamma ray \textit{coincidence} analysis was used.
This relies on the fact that gamma rays that are recorded with a small enough separation in time (typically $\sim$ns) can be attributed to the decay of an individual nucleus, if enough of these \textit{coincident} detections are made.
Finding new gamma rays that are coincident with those previously concluded (either in this work or preceding studies) to be emitted by $\ce{^{148}Ce}$ allowed the existing level scheme to be expanded upon.

\medskip
Where possible, the spin-parities of observed excited states were then inferred. Consideration of the relative intensities of and total internal conversion coefficients for different gamma ray transitions enabled this. The latter coefficients are given by the intensity ratio of events where a specific excited state emits an electron to the total intensity of transitions from that state.

\section*{Contribution}
%Describe their results and conclusions, assess how they fit into the field. Mention any important outcomes or future directions enabled by their work.
20 new transitions and nine new levels were added to the level scheme of $\ce{^{148}Ce}$. Spins and parities were assigned (tentatively in some cases) for some of the newly discovered states and either confirmed or redefined for some other existing levels. Internal conversion coefficients for some transitions were also measured.

\medskip
By analysing the differences in energies between certain rotational bands of opposite parity, the observed octupole correlations in $\ce{^{148}Ce}$ were suggested to be similar to those found in $\ce{^{144,146}Ce}$.
These bands are groups of distinct energy levels that correspond to a certain kind of rotational (or vibrational) motion displayed by a nucleus.
However, it was proposed that the different bands within $\ce{^{148}Ce}$ may exhibit octupole correlations with different strengths.
This conclusion was reinforced by further analysis of the relative transition energies and intensities within each rotational band. 

\medskip
Yet, after analysing the moment of inertia as a function of the rotational energy of the nucleus, the authors suggested that the deformation of this nucleus due to these octupole correlations may be unstable, prompting further theoretical study. 

\section*{Relevance}
%Explain how this paper is relevant to your project and what you are trying to get out of the paper. Does it suggest other papers you should read, or give you new information you need?
While access to ANU's experimental facilities remains heavily restricted due to the COVID-19 pandemic, I will be analysing data from an experiment performed to gauge octupole collectivity in the neutron-rich nuclides $\ce{^{150,152}Ce}$.
These are located near $Z=56$ and $N=88$ but remain unstudied.
As such, having an understanding of what octupole correlations are and how they can be investigated is vital.

\section*{Quality}
%Discuss the quality of the writing and presentation of results, and try to assess how much you trust their assumptions, approach, and conclusions. You can talk to a supervisor to help you with this.
This work was written clearly and the analysis was detailed, with the development of their final level scheme via gamma ray spectroscopic analysis described in detail. 
However, to understand some parts of the paper, I did have to refer to some other sources.

\medskip
Yet, these authors provided multiple pieces of evidence for their conclusion that they did observe octupole correlations in $\ce{^{148}Ce}$, making their conclusions trustworthy. Seeing explicit discussion of how they reached these conclusions was also worthwhile, given I will be performing a similar analysis for my thesis.

\section*{Questions/Directions}
%List any questions the paper has raised, and how you might go about addressing them. Describe any directions your reading/own work could take that are suggested by this paper.
After reading this paper, I should do the following.
\begin{itemize}
\item Brush up on my knowledge of the finer details of nuclear structure, by referring to my notes from PHYS3105 and relevant textbooks.
\item Explore the references of this paper to better understand the theoretical models that predict the existence of these octupole correlations.
\end{itemize}

%\vspace*{-\baselineskip}
%\bibliographystyle{ieeetr}
%\bibliography{references.bib}{}

%Submission details
% So the resubmission assignment is available on Wattle. It is a two-part assignment - submit the feedback file you received in part 2, the actual final journal entry in part 1.	


\end{document}