\documentclass[12pt,a4paper]{article}
\usepackage[utf8]{inputenc}
%\usepackage[hidelinks]{hyperref}
\usepackage{hyperref}
\usepackage{xcolor}
\usepackage{cite}
\usepackage[version=4]{mhchem} % for isotope symbols
\usepackage{a4wide}
\usepackage[lmargin=1cm,rmargin=1cm,tmargin=1cm,bmargin=2cm,headheight=0in]{geometry}
\usepackage{amssymb}


\hypersetup{colorlinks, linkcolor={blue!50!black}, citecolor={blue!50!black},
	urlcolor={blue!50!black}}

\begin{document}
\title{Journal Entry - Week Five - Entry Four - Draft}
\author{Caspian Nicholls, u1027945}
\date{\today}

\maketitle

\noindent
I give permission for portions of this work to be used as examples in future science communication course notes.
% \textit{[Delete if you don't want to do this]}\\

\medskip
\noindent
\textbf{Article title:} Search for octupole correlations in neutron-rich $\ce{^{148}Ce}$ nucleus \\
\textbf{Authors:} Y. J. Chen, S. J. Zhu, J. H. Hamilton, A. V. Ramayya, J. K. Hwang, M. Sakhaee, Y. X. Luo, J. O. Rasmussen, K. Li, I. Y. Lee, X. L. Che, H. B. Ding, and M. L. Li \\
\textbf{Reference:} Phys. Rev. C 73, 054316, May 2006. \\
\textbf{Link:}  \href{10.1103/PhysRevC.73.054316}{Article abstract}. \\
\textbf{Word count:} 707
%\textit{Give the word count of everything below this point, e.g. copy paste into the \href{https://app.uio.no/ifi/texcount/online.php}{TeXcount} online tool. 750 word limit.}

\section*{Context}
%Briefly summarise the background of the area of physics relevant to the problem they are solving. Why it is an important or relevant area and problem?
Theoretical calculations that make use of accepted models of the structure of nuclei predict that several nuclei with proton numbers $Z$ around 56 and neutron numbers $N$ around 88 should exhibit octupole correlations. These correlations occur most strongly when there is a strong interaction between nuclear orbitals that for principal, orbital angular momentum and total angular momentum quantum numbers $n$, $l$ and $j$, have differences $\Delta n = 1$ and $\Delta l = \Delta j = 3$ due to the octupole interaction component of the nuclear Hamiltonian having significant magnitude. The observation of octupole correlations in nuclei is evidence that the charge distribution of the nucleus may exhibit octupole deformation, so that its shape is no longer spherical. Prior to this paper, there remained nuclei near $Z = 56$ and $N = 88$ for which experimental evidence of octupole correlations had not been recorded. Ultimately, observation of these correlations can provide useful information about the structure of these nuclei.


\section*{Purpose}
%Explain what problem they are solving, or what information they were trying to gain.
In this work, the authors perform experiments to hunt for signs of octupole correlation in $\ce{^{148}Ce}$, which has $Z = 58$ and $N = 90$, so it is one such nuclei that should exhibit this type of correlation. They also seek to extend the level scheme of this nucleus.


\section*{Approach}
%Describe explicitly the method they use to complete their work, i.e. the experimental system and measurements, the model design and computational techniques used, or the setup of their equations and the assumptions applied.
Gamma rays that were emitted from the excited primary fission fragments (which included excited forms of $\ce{^{148}Ce}$) of the spontaneous fission of $\ce{^{252}Cf}$ were measured to study $\ce{^{148}Ce}$. By analysing groups of three gamma rays that were emitted within a narrow enough time window to be confident that they originated from the same nucleus, the relative intensities of different gamma ray transitions and the total internal coefficients, % why does this last one matter?
new transitions and levels were identified in $\ce{^{148}Ce}$.

\section*{Contribution}
%Describe their results and conclusions, assess how they fit into the field. Mention any important outcomes or future directions enabled by their work.
20 new transitions and nine new levels were added to the energy level structure of $\ce{^{148}Ce}$ as a result of their analysis. Spins and parities were able to be assigned (tentatively in some cases) for some of the newly discovered states and either confirmed or refined for other existing levels. Internal conversion coefficients (where a high-energy electron is emitted instead of a gamma ray) for some transitions were also measured. 

\medskip
Analysis of the differences in energies between bands of opposite parity (positive and negative) led the authors to suggest that the octupole correlation they observed in $\ce{^{148}Ce}$ has similarities to that observed within the nearby isotopes $\ce{^{144,146}Ce}$. However, they found that different bands of energy levels within $\ce{^{148}Ce}$ (which relate to energy states of the nucleus that arise when it is rotated, each with different rotational symmetries) may have octupole correlations with different strengths. This conclusion was also supported by further analysis of the relative transition energies and intensities within each rotational band. 

\medskip
However, analysis of the moment of inertia as a function of the rotational energy of the nucleus led the authors to suggest that the octupole deformation seen in this nucleus may be unstable, prompting further theoretical study. 

\section*{Relevance}
%Explain how this paper is relevant to your project and what you are trying to get out of the paper. Does it suggest other papers you should read, or give you new information you need?
Following the disruptions caused by the COVID-19 pandemic, we have had to come up with an alternate path for my Honours project to pursue whilst we are not permitted to access ANU's experimental facilities. This back-up plan involves the analysis of data generated from an experiment that was performed with the aim of gauging octupole collectivity in the neutron-rich nuclei $\ce{^{148,150,152}Ce}$. As such, having an understanding of what octupole correlations are and how they are studied is vital.

\section*{Quality}
%Discuss the quality of the writing and presentation of results, and try to assess how much you trust their assumptions, approach, and conclusions. You can talk to a supervisor to help you with this.
This work was clearly written and the analysis was thorough, as the authors describe in detail how they developed their final level scheme via gamma ray spectroscopic analysis of their measured gamma ray spectra. I did have to refer to some other papers to clarify parts of the paper that I was not familiar with, but fortunately clear and precise explanations of these details were easy to come by. These authors also provided multiple pieces of evidence for their conclusion that they did observe octupole correlations in $\ce{^{148}Ce}$, increasing the trustworthiness of their work.

\section*{Questions/Directions}
%List any questions the paper has raised, and how you might go about addressing them. Describe any directions your reading/own work could take that are suggested by this paper.
After reading this paper, it is clear to me that I need to brush up on my knowledge of the finer details of nuclear structure. The best way to do this would be referring back to the notes I have made from a course I took last year on this subject at ANU. Before I start to analyse the experimental data I referred to in the \textit{Relevance} section, reading more deeply about the theoretical models that predict these octupole correlations (starting from the references in this paper) would also be worthwhile.

\vspace*{-\baselineskip}
\bibliographystyle{ieeetr}
\bibliography{references.bib}{}


\end{document}